\chapter{Introduction}
\label{introduction}

\section{Background}
\subsection{SONIC Project}
Today’s communication happens mostly through Online Social Network (OSN), but with their proliferation, some problems have arisen. The main issue is related to the fact that OSN are built in a closed an proprietary manner (think of Facebook, Twitter or LinkedIn) and don’t allow a smooth  communication among them. Therefore, the user is forced to create different accounts, and build different networks within for every platform. All this accounts and information correspond to the same user, but are heavily segregated from one another. This segregation is caused by the platform which aim to bind the user with a lock-in effect. 

The SONIC project has been proposed by the team working for the Telekom Innovation Laboratories [clarify this] to overcome this problem and facilitate a seamless connectivity between different OSN [reference 1] allowing the migration of accounts between different platforms. 

The project is supported by a distributed and domain-independent ID management architecture where a GSLS (Global Social Lookup System) is employed to map Global ID to URL of a social profile. This mapping is possible thanks to datasets called Social Records which are digitally signed. 

\subsection{Blockchain}

\subsection{Identity Management}
