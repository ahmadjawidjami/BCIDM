\chapter{Introduction}
\label{introduction}

\section{Background}
\subsection{SONIC Project}
Today’s communication happens mostly through Online Social Network (OSN), but with their proliferation, some problems have arisen. The main issue is related to the fact that OSN are built in a closed an proprietary manner (think of Facebook, Twitter or LinkedIn) and don’t allow a smooth  communication among them. Therefore, the user is forced to create different accounts, and build different networks within for every platform. All this accounts and information correspond to the same user, but are heavily segregated from one another. This segregation is caused by the platform which aim to bind the user with a lock-in effect. 

The SONIC project has been proposed by the team working for the Telekom Innovation Laboratories [clarify this] to overcome this problem and facilitate a seamless connectivity between different OSN [reference 1] allowing the migration of accounts between different platforms. 

The project is supported by a distributed and domain-independent ID management architecture where a GSLS (Global Social Lookup System) is employed to map Global ID to URL of a social profile. This mapping is possible thanks to datasets called Social Records which are digitally signed. 

\subsection{Blockchain}
Blockchain is a distributed ledger technology used to keep track of Bitcoin cryptocurrency transactions. Distributed ledgers create a data structure – like a chain – where records of every single Bitcoin transaction live. To prevent “double spend,” all Bitcoin transactions are validated and then permanently archived in the cryptographic ledger or chain. The validation is done via a peer-to-peer process that is hugely computer-intensive. It is supported by a global network of volunteers – known as “miners” – who are incentivized mainly by Bitcoin’s mining reward.

In essence, Bitcoin uses cryptography to enable participants on the network to update the ledger in a secure way without the need for a central authority. The key to Blockchain was to agree on the order of entries in the ledger. Once this was in place, distributed control of Bitcoin was possible.

\subsection{Identity Management}
Identity management refers to the process of employing emerging technologies to manage information about the identity of users and control access to company resources [1]. The goal of identity management is to improve productivity and security while lowering costs associated with managing users and their identities, attributes, and credentials.

Everyone is using internet to interact with different digital service platforms.IT have many forms like accessing social sites, online shopping services ,and interacting with your online banking account . Interactions with these service providers require that each user has digital identity so that user is authorized to access digital platform. There are certain reasons digital platform uses identities and storing all the information related to the user to grow their business and improve user experiences, and defend against certain attacks externally as well as to take care of the privacy of the user.

Identities were used in different ways like accessing personal computer we use only username and password .Digital platforms on the other hand like Facebook, yahoo use domain name with combination of username to differentiate different users because these platforms have many users so domain name can make username uniquely identifiable [2]. Anonymous credentials as another way of providing authentication
assertions that don’t reveal the user’s identity to a service provider for example, U prove technology implemented by Microsoft.

The problem in these digital platforms is the absence of federated directories. Microsoft defines federation as “the technology and business arrangements necessary for the interconnecting of users, applications, and systems. This includes
authentication, distributed processing and storage, data sharing, and more.”

Federated directories interact and trust each other, thus allowing secure information sharing between applications. Companies are currently running isolated, independent directories that neither interact with nor trust each other.
This is a result of applications having their own proprietary identity stores. Each
proprietary directory requires its own method of user administration, user
provisioning, and user access control. The problem with proprietary identity stores is that users require login for every application, which in turn burdens users with having to remember numerous username and password combinations.

The need for blockchain based identity management is particularly noticeable in the internet age,we have faced identity management challenges since the dawn of the Internet. Prime among them: security, privacy, and usability.
Blockchain technology may offer a way to circumvent these problems by delivering a secure solution without the need for a trusted, central authority[3]. It can be used for creating an identity on the blockchain, making it easier to manage for individuals, giving them greater control over has personal information and how they access it.


