\documentclass[10pt]{article}
\usepackage[usenames]{color} %used for font color
\usepackage{amssymb} %maths
\usepackage{amsmath} %maths
\usepackage[utf8]{inputenc} %useful to type directly diacritic characters
\begin{document}
\[\chapter{Evaluation}
\label{evaluation}

\section{Performance}

\subsection{Get}
The get operations are nearly instantly, since the state of the blockchain does not need to be changed.
The get functionalities are therefore very similar to the DTH.

\subsection{Update}
The time needed in order to update one social record is not predictable, could be almost instant (if it is directly inserted in a block) or never being mined.
One of the key factors in order to have the transaction mined is to increase the transaction price.
In fact, miners will be willing to put expensive transactions in their block, aiming to maximise the rewards of the proof of work.
An expensive transaction is a transaction with an high \textit{gasPrice}, meaning that who build the transaction is prone to pay one particular price per each gas.
Another factor is the \textit{gasLimit}, saying that the total theoretical price for one transaction is given by \textit{maxPrice = gasPrice × gasLimit}.
In the reality, the unused \textit{gasLimit} is returned to the creator of the transaction.
For this reason and for the constant changes that are affecting Ethereum, it has not any validity to create a benchmark trying to find an average updating time \cite{miningTime}.
It is nevertheless reasonable to estimate it in 1-3 minutes.

\section{Security}

If on the performance side this new approach may appear not optimal, the security is strongly enhanced in compare with the previous DHT implementation.

It has been proved in the concept section the security of this approach, in particular:

\begin{list}{}{}
\item {Bigger Node}: Since the social record is stored in one of the main public blockchains, it is very unlikely to perpetrate an attack aiming at the 50\%+1 of the network (proof of immunity).
\item {Contract}: The contract has a consistent number of checkers in order to stop attack to change one particular social record (functionalities that were not available within the DHT.
\item {Client Server}: Not only the storying method is secure but also the communication between the client and the server since the transactions are sent already encrypted.
\item {Malicious GSLS}: in this approach a malicious GSLS server cannot change the content of one social record nor upload an old one (due to the expired nonce of the transaction). The only feasible attack is to send false information to the client (e.g. nonce), resulting in an invalid transaction. Nevertheless the open source nature of the sonic project allows to arbitrary choose the GSLS server the user wants to link to. It will be therefore only necessary to change the server location in the desktop application settings.
\end{list}\]
\end{document}